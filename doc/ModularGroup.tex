% generated by GAPDoc2LaTeX from XML source (Frank Luebeck)
\documentclass[a4paper,11pt]{report}

\usepackage{a4wide}
\sloppy
\pagestyle{myheadings}
\usepackage{amssymb}
\usepackage[utf8]{inputenc}
\usepackage{makeidx}
\makeindex
\usepackage{color}
\definecolor{FireBrick}{rgb}{0.5812,0.0074,0.0083}
\definecolor{RoyalBlue}{rgb}{0.0236,0.0894,0.6179}
\definecolor{RoyalGreen}{rgb}{0.0236,0.6179,0.0894}
\definecolor{RoyalRed}{rgb}{0.6179,0.0236,0.0894}
\definecolor{LightBlue}{rgb}{0.8544,0.9511,1.0000}
\definecolor{Black}{rgb}{0.0,0.0,0.0}

\definecolor{linkColor}{rgb}{0.0,0.0,0.554}
\definecolor{citeColor}{rgb}{0.0,0.0,0.554}
\definecolor{fileColor}{rgb}{0.0,0.0,0.554}
\definecolor{urlColor}{rgb}{0.0,0.0,0.554}
\definecolor{promptColor}{rgb}{0.0,0.0,0.589}
\definecolor{brkpromptColor}{rgb}{0.589,0.0,0.0}
\definecolor{gapinputColor}{rgb}{0.589,0.0,0.0}
\definecolor{gapoutputColor}{rgb}{0.0,0.0,0.0}

%%  for a long time these were red and blue by default,
%%  now black, but keep variables to overwrite
\definecolor{FuncColor}{rgb}{0.0,0.0,0.0}
%% strange name because of pdflatex bug:
\definecolor{Chapter }{rgb}{0.0,0.0,0.0}
\definecolor{DarkOlive}{rgb}{0.1047,0.2412,0.0064}


\usepackage{fancyvrb}

\usepackage{mathptmx,helvet}
\usepackage[T1]{fontenc}
\usepackage{textcomp}


\usepackage[
            pdftex=true,
            bookmarks=true,        
            a4paper=true,
            pdftitle={Written with GAPDoc},
            pdfcreator={LaTeX with hyperref package / GAPDoc},
            colorlinks=true,
            backref=page,
            breaklinks=true,
            linkcolor=linkColor,
            citecolor=citeColor,
            filecolor=fileColor,
            urlcolor=urlColor,
            pdfpagemode={UseNone}, 
           ]{hyperref}

\newcommand{\maintitlesize}{\fontsize{50}{55}\selectfont}

% write page numbers to a .pnr log file for online help
\newwrite\pagenrlog
\immediate\openout\pagenrlog =\jobname.pnr
\immediate\write\pagenrlog{PAGENRS := [}
\newcommand{\logpage}[1]{\protect\write\pagenrlog{#1, \thepage,}}
%% were never documented, give conflicts with some additional packages

\newcommand{\GAP}{\textsf{GAP}}

%% nicer description environments, allows long labels
\usepackage{enumitem}
\setdescription{style=nextline}

%% depth of toc
\setcounter{tocdepth}{1}





%% command for ColorPrompt style examples
\newcommand{\gapprompt}[1]{\color{promptColor}{\bfseries #1}}
\newcommand{\gapbrkprompt}[1]{\color{brkpromptColor}{\bfseries #1}}
\newcommand{\gapinput}[1]{\color{gapinputColor}{#1}}


\begin{document}

\logpage{[ 0, 0, 0 ]}
\begin{titlepage}
\mbox{}\vfill

\begin{center}{\maintitlesize \textbf{ ModularGroup \mbox{}}}\\
\vfill

\hypersetup{pdftitle= ModularGroup }
\markright{\scriptsize \mbox{}\hfill  ModularGroup  \hfill\mbox{}}
{\Huge \textbf{ Computing with finite-index subgroups of SL(2,Z) \mbox{}}}\\
\vfill

{\Huge  0.0.1 \mbox{}}\\[1cm]
{ 10 January 2018 \mbox{}}\\[1cm]
\mbox{}\\[2cm]
{\Large \textbf{ Max Mustermann\\
    \mbox{}}}\\
{\Large \textbf{ John Doe\\
    \mbox{}}}\\
\hypersetup{pdfauthor= Max Mustermann\\
    ;  John Doe\\
    }
\end{center}\vfill

\mbox{}\\
{\mbox{}\\
\small \noindent \textbf{ Max Mustermann\\
    }  Email: \href{mailto://max.mustermann@example.com} {\texttt{max.mustermann@example.com}}\\
  Homepage: \href{http://www.math.uni-sb.de/ag/weitze/} {\texttt{http://www.math.uni-sb.de/ag/weitze/}}\\
  Address: \begin{minipage}[t]{8cm}\noindent
 AG Weitze-Schmith{\"u}sen\\
 FR 6.1 Mathematik\\
 Universit{\"a}t des Saarlandes\\
 D-66041 Saarbr{\"u}cken\\
 \end{minipage}
}\\
{\mbox{}\\
\small \noindent \textbf{ John Doe\\
    }  Email: \href{mailto://john.doe@example.com} {\texttt{john.doe@example.com}}\\
  Homepage: \href{http://www.math.uni-sb.de/ag/weitze/} {\texttt{http://www.math.uni-sb.de/ag/weitze/}}\\
  Address: \begin{minipage}[t]{8cm}\noindent
 AG Weitze-Schmith{\"u}sen\\
 FR 6.1 Mathematik\\
 Universit{\"a}t des Saarlandes\\
 D-66041 Saarbr{\"u}cken\\
 \end{minipage}
}\\
\end{titlepage}

\newpage\setcounter{page}{2}
\newpage

\def\contentsname{Contents\logpage{[ 0, 0, 1 ]}}

\tableofcontents
\newpage

     
\chapter{\textcolor{Chapter }{The Modular Group and its subgroups}}\label{Chapter_The_Modular_Group_and_its_subgroups}
\logpage{[ 1, 0, 0 ]}
\hyperdef{L}{X7CFC464D875B9922}{}
{
  This package contains methods for computing with finite-index subgroups of the
modular group $\mathrm{SL}(2, \mathbb{Z})$ which are given by a coset permutation representation with respect to the
generators 
$$ S = \left( {\begin{array}{cc} 0 & -1 \\ 1 & 0 \\ \end{array} } \right) \quad T = \left( {\begin{array}{cc} 1 & 1 \\ 0 & 1 \\ \end{array} } \right) $$
 We will call these subgroups 'modular subgroups'. 
\section{\textcolor{Chapter }{Construction of modular subgroups}}\label{Chapter_The_Modular_Group_and_its_subgroups_Section_Construction_of_modular_subgroups}
\logpage{[ 1, 1, 0 ]}
\hyperdef{L}{X82B1267A8369ABC3}{}
{
  In this section we describe how to construct modular subgroups from a given
coset permutation representation or from a list of generator matrices and some
related methods. 

\subsection{\textcolor{Chapter }{ModularSubgroup (for two permutations)}}
\logpage{[ 1, 1, 1 ]}\nobreak
\hyperdef{L}{X874D675284A86159}{}
{\noindent\textcolor{FuncColor}{$\triangleright$\enspace\texttt{ModularSubgroup({\mdseries\slshape s, t})\index{ModularSubgroup@\texttt{ModularSubgroup}!for two permutations}
\label{ModularSubgroup:for two permutations}
}\hfill{\scriptsize (function)}}\\
\textbf{\indent Returns:\ }
an object representing a modular subgroup 



 This method constructs a modular subgroup from two given permutations
(provided they describe a coset action). }

 

\subsection{\textcolor{Chapter }{ModularSubgroup (for a list of matrices)}}
\logpage{[ 1, 1, 2 ]}\nobreak
\hyperdef{L}{X86C13616837F567D}{}
{\noindent\textcolor{FuncColor}{$\triangleright$\enspace\texttt{ModularSubgroup({\mdseries\slshape gens})\index{ModularSubgroup@\texttt{ModularSubgroup}!for a list of matrices}
\label{ModularSubgroup:for a list of matrices}
}\hfill{\scriptsize (function)}}\\
\textbf{\indent Returns:\ }
an object representing a modular subgroup 



 This is another constructor for a modular subgroup, with the difference that
it takes a list of generator matrices as input and calculates the coset graph
of the generated group. One has to be careful though when using this method,
because no check is performed if the generated group has finite index!
Internally, when trying to calculate the coset graph, we just enumerate all
cosets until we are done or some limit fixed is reached. This also exposes
another weakness of this method: The coset enumeration might be very
time-consuming, so constructing modular subgroups from a list of generators is
not always feasible. On the other hand, the clear advantage of constructing a
modular subgroup in this way is that it will alreasy know its generators. So
future computations with this group involving generators will most likely be
faster. }

 

\subsection{\textcolor{Chapter }{DefinesCosetAction (for two permutations)}}
\logpage{[ 1, 1, 3 ]}\nobreak
\hyperdef{L}{X811EE02082291F48}{}
{\noindent\textcolor{FuncColor}{$\triangleright$\enspace\texttt{DefinesCosetAction({\mdseries\slshape s, t})\index{DefinesCosetAction@\texttt{DefinesCosetAction}!for two permutations}
\label{DefinesCosetAction:for two permutations}
}\hfill{\scriptsize (function)}}\\
\textbf{\indent Returns:\ }
true or false 



 This is an auxiliary method that takes two permutations as input and checks if
they describe the action of the generators $S$ and $T$ on the cosets of some group. This check is for example performed in the
constructor for a modular subgroup. }

 

\subsection{\textcolor{Chapter }{CosetActionFromGenerators (for a list of generator matrices)}}
\logpage{[ 1, 1, 4 ]}\nobreak
\hyperdef{L}{X7954F4F283B4F8C3}{}
{\noindent\textcolor{FuncColor}{$\triangleright$\enspace\texttt{CosetActionFromGenerators({\mdseries\slshape gens})\index{CosetActionFromGenerators@\texttt{CosetActionFromGenerators}!for a list of generator matrices}
\label{CosetActionFromGenerators:for a list of generator matrices}
}\hfill{\scriptsize (function)}}\\
\textbf{\indent Returns:\ }
two permutations 



 Takes a list of generator matrices and calculates the coset permutation
representation of the generated subgroup. The same warning as above applies:
No check is performed if the generated subgroup actually has finite index. }

 

\subsection{\textcolor{Chapter }{STDecomposition (for a matrix in SL(2,Z))}}
\logpage{[ 1, 1, 5 ]}\nobreak
\hyperdef{L}{X7C82B9BE807EF3A8}{}
{\noindent\textcolor{FuncColor}{$\triangleright$\enspace\texttt{STDecomposition({\mdseries\slshape M})\index{STDecomposition@\texttt{STDecomposition}!for a matrix in SL(2,Z)}
\label{STDecomposition:for a matrix in SL(2,Z)}
}\hfill{\scriptsize (function)}}\\
\textbf{\indent Returns:\ }
a word in $S$ and $T$ 



 Takes a matrix in $\mathrm{SL}(2, \mathbb{Z})$ and decomposes it in a word in the generators $S$ and $T$. }

 

\subsection{\textcolor{Chapter }{SAction (for a modular subgroup)}}
\logpage{[ 1, 1, 6 ]}\nobreak
\hyperdef{L}{X816B8B0084130418}{}
{\noindent\textcolor{FuncColor}{$\triangleright$\enspace\texttt{SAction({\mdseries\slshape G})\index{SAction@\texttt{SAction}!for a modular subgroup}
\label{SAction:for a modular subgroup}
}\hfill{\scriptsize (function)}}\\
\textbf{\indent Returns:\ }
a permutation 



 Takes a modular subgroup and returns a permutation corresponding to the action
of the generator matrix $S$. }

 

\subsection{\textcolor{Chapter }{TAction (for a modular subgroup)}}
\logpage{[ 1, 1, 7 ]}\nobreak
\hyperdef{L}{X7A39A1937DF54D42}{}
{\noindent\textcolor{FuncColor}{$\triangleright$\enspace\texttt{TAction({\mdseries\slshape G})\index{TAction@\texttt{TAction}!for a modular subgroup}
\label{TAction:for a modular subgroup}
}\hfill{\scriptsize (function)}}\\
\textbf{\indent Returns:\ }
a permutation 



 Takes a modular subgroup and returns a permutation corresponding to the action
of the generator matrix $T$. }

 }

 
\section{\textcolor{Chapter }{Computations with modular subgroups}}\label{Chapter_The_Modular_Group_and_its_subgroups_Section_Computations_with_modular_subgroups}
\logpage{[ 1, 2, 0 ]}
\hyperdef{L}{X85234B2885D8DF56}{}
{
  In this section we describe the implemented method for computing with modular
subgroups. 

\subsection{\textcolor{Chapter }{Index (for a modular subgroup)}}
\logpage{[ 1, 2, 1 ]}\nobreak
\hyperdef{L}{X80E1728B78D0F26D}{}
{\noindent\textcolor{FuncColor}{$\triangleright$\enspace\texttt{Index({\mdseries\slshape G})\index{Index@\texttt{Index}!for a modular subgroup}
\label{Index:for a modular subgroup}
}\hfill{\scriptsize (function)}}\\
\textbf{\indent Returns:\ }
a natural number 



 Takes a modular subgroup and returns its index in $\mathrm{SL}(2, \mathbb{Z})$. }

 

\subsection{\textcolor{Chapter }{IsCongruenceSubgroup (for a modular subgroup)}}
\logpage{[ 1, 2, 2 ]}\nobreak
\hyperdef{L}{X7819EFB17A4E722C}{}
{\noindent\textcolor{FuncColor}{$\triangleright$\enspace\texttt{IsCongruenceSubgroup({\mdseries\slshape G})\index{IsCongruenceSubgroup@\texttt{IsCongruenceSubgroup}!for a modular subgroup}
\label{IsCongruenceSubgroup:for a modular subgroup}
}\hfill{\scriptsize (function)}}\\
\textbf{\indent Returns:\ }
true or false 



 Tests whether a given modular subgroup is a congruence subgroup. }

 

\subsection{\textcolor{Chapter }{RightCosetRepresentatives (for a modular subgroup)}}
\logpage{[ 1, 2, 3 ]}\nobreak
\hyperdef{L}{X787BDFA47BAA14DA}{}
{\noindent\textcolor{FuncColor}{$\triangleright$\enspace\texttt{RightCosetRepresentatives({\mdseries\slshape G})\index{RightCosetRepresentatives@\texttt{RightCosetRepresentatives}!for a modular subgroup}
\label{RightCosetRepresentatives:for a modular subgroup}
}\hfill{\scriptsize (function)}}\\
\textbf{\indent Returns:\ }
a list of matrices 



 Calculates a list of representatives of the right cosets of a given modular
subgroup. }

 

\subsection{\textcolor{Chapter }{GeneralizedLevel (for a modular subgroup)}}
\logpage{[ 1, 2, 4 ]}\nobreak
\hyperdef{L}{X87E645C67DE537C4}{}
{\noindent\textcolor{FuncColor}{$\triangleright$\enspace\texttt{GeneralizedLevel({\mdseries\slshape G})\index{GeneralizedLevel@\texttt{GeneralizedLevel}!for a modular subgroup}
\label{GeneralizedLevel:for a modular subgroup}
}\hfill{\scriptsize (function)}}\\
\textbf{\indent Returns:\ }
a natural number 



 Computes the generalized level (i.e. the lowest common multiple of all cusp
widths) of a given modular subgroup. }

 

\subsection{\textcolor{Chapter }{GeneratorsOfGroup (for a modular subgroup)}}
\logpage{[ 1, 2, 5 ]}\nobreak
\hyperdef{L}{X83C268CB8293911B}{}
{\noindent\textcolor{FuncColor}{$\triangleright$\enspace\texttt{GeneratorsOfGroup({\mdseries\slshape G})\index{GeneratorsOfGroup@\texttt{GeneratorsOfGroup}!for a modular subgroup}
\label{GeneratorsOfGroup:for a modular subgroup}
}\hfill{\scriptsize (function)}}\\
\textbf{\indent Returns:\ }




 Calculates a list of generators for a given modular subgroup. Note: The
returned list might contain redundant generators (or even duplicates). This
calculation involves enumerating the cosets of the given group and might
become very slow for large index. }

 

\subsection{\textcolor{Chapter }{IsElementOf (for a matrix in SL(2,Z) and a modular subgroup)}}
\logpage{[ 1, 2, 6 ]}\nobreak
\hyperdef{L}{X84976AF584CEF44C}{}
{\noindent\textcolor{FuncColor}{$\triangleright$\enspace\texttt{IsElementOf({\mdseries\slshape A, G})\index{IsElementOf@\texttt{IsElementOf}!for a matrix in SL(2,Z) and a modular subgroup}
\label{IsElementOf:for a matrix in SL(2,Z) and a modular subgroup}
}\hfill{\scriptsize (function)}}\\
\textbf{\indent Returns:\ }
true or false 



 This is a membership test for modular subgroups given by a coset permutation
representation. }

 

\subsection{\textcolor{Chapter }{CuspWidth (for a rational number or infinity and a modular subgroup)}}
\logpage{[ 1, 2, 7 ]}\nobreak
\hyperdef{L}{X7D04D43078672A95}{}
{\noindent\textcolor{FuncColor}{$\triangleright$\enspace\texttt{CuspWidth({\mdseries\slshape c, G})\index{CuspWidth@\texttt{CuspWidth}!for a rational number or infinity and a modular subgroup}
\label{CuspWidth:for a rational number or infinity and a modular subgroup}
}\hfill{\scriptsize (function)}}\\
\textbf{\indent Returns:\ }
a natural number 



 Calculates the width of $c$ with respect to a given modular subgroup, i.e. the smallest $k$ such that $\pm gT^{k}g^{-1} \in G$ where $g \in \mathrm{SL}(2, \mathbb{Z})$ such that $g\infty = c$. }

 

\subsection{\textcolor{Chapter }{CuspsEquivalent (for two cusps (i.e. rational numbers or infinity) and a modular subgroup)}}
\logpage{[ 1, 2, 8 ]}\nobreak
\hyperdef{L}{X7CB9C6567C1293CF}{}
{\noindent\textcolor{FuncColor}{$\triangleright$\enspace\texttt{CuspsEquivalent({\mdseries\slshape c1, c2, G})\index{CuspsEquivalent@\texttt{CuspsEquivalent}!for two cusps (i.e. rational numbers or infinity) and a modular subgroup}
\label{CuspsEquivalent:for two cusps (i.e. rational numbers or infinity) and a modular subgroup}
}\hfill{\scriptsize (function)}}\\
\textbf{\indent Returns:\ }
true or false 



 Checks if two cusps are equivalent with respect to a given modular subgroup. }

 

\subsection{\textcolor{Chapter }{Cusps (for a modular subgroup)}}
\logpage{[ 1, 2, 9 ]}\nobreak
\hyperdef{L}{X7D1122227CD92271}{}
{\noindent\textcolor{FuncColor}{$\triangleright$\enspace\texttt{Cusps({\mdseries\slshape G})\index{Cusps@\texttt{Cusps}!for a modular subgroup}
\label{Cusps:for a modular subgroup}
}\hfill{\scriptsize (function)}}\\
\textbf{\indent Returns:\ }
a list of cusps 



 Calculates a list of inequivalent cusp representative for a given modular
subgroup. }

 

\subsection{\textcolor{Chapter }{IndexModN (for a modular subgroup and a natural number > 1)}}
\logpage{[ 1, 2, 10 ]}\nobreak
\hyperdef{L}{X80236AC5801ACD53}{}
{\noindent\textcolor{FuncColor}{$\triangleright$\enspace\texttt{IndexModN({\mdseries\slshape G, N})\index{IndexModN@\texttt{IndexModN}!for a modular subgroup and a natural number > 1}
\label{IndexModN:for a modular subgroup and a natural number > 1}
}\hfill{\scriptsize (function)}}\\
\textbf{\indent Returns:\ }
a natural number 



 This method computes the index of the image of $G$ in $\mathrm{SL}(2, \mathbb{Z}/N\mathbb{Z})$ under the projection 
\[\pi_N : \mathrm{SL}(2,\mathbb{Z}) \rightarrow \mathrm{SL}(2,
\mathbb{Z}/N\mathbb{Z})\]
 }

 

\subsection{\textcolor{Chapter }{Deficiency (for a modular subgroup and a natural number > 1)}}
\logpage{[ 1, 2, 11 ]}\nobreak
\hyperdef{L}{X7B97126F7CC34C75}{}
{\noindent\textcolor{FuncColor}{$\triangleright$\enspace\texttt{Deficiency({\mdseries\slshape G, N})\index{Deficiency@\texttt{Deficiency}!for a modular subgroup and a natural number > 1}
\label{Deficiency:for a modular subgroup and a natural number > 1}
}\hfill{\scriptsize (function)}}\\
\textbf{\indent Returns:\ }
a natural number 



 This method calculates the so-called deficiency $f_N$ of a modular subgroup, i.e. the index $[ \Gamma(N) : \Gamma(N) \cap G ]$. }

 }

 }

 \def\indexname{Index\logpage{[ "Ind", 0, 0 ]}
\hyperdef{L}{X83A0356F839C696F}{}
}

\cleardoublepage
\phantomsection
\addcontentsline{toc}{chapter}{Index}


\printindex

\immediate\write\pagenrlog{["Ind", 0, 0], \arabic{page},}
\newpage
\immediate\write\pagenrlog{["End"], \arabic{page}];}
\immediate\closeout\pagenrlog
\end{document}
